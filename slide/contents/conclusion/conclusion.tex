\section{Conclusion}
\begin{frame}{Conclusion}
    \begin{block}{Conclusion}
        The prototype pattern allows us to easily let objects \textbf{access} and \textbf{inherit} properties from other objects. Since the prototype chain allows us to access properties that aren't directly defined on the object itself, we can avoid \textbf{duplication} of methods and properties, thus \textbf{reducing} the amount of memory used  \cite{patterns_prototype}.
    \end{block}
\end{frame}

\section{Conclusion}
\begin{frame}{Conclusion}
    \begin{block}{Conclusion}
        The prototype pattern allows us to easily let objects \textbf{access} and \textbf{inherit} properties from other objects. Since the prototype chain allows us to access properties that aren't directly defined on the object itself, we can avoid \textbf{duplication} of methods and properties, thus \textbf{reducing} the amount of memory used  \cite{patterns_prototype}.
    \end{block}

    \begin{block}{Warning}
        While the Prototype pattern offers several benefits, it also comes with challenges related to \textbf{deep copying, memory usage} and \textbf{potential misuse}. But if you put it in the right scenarios, the Prototype pattern can be a valuable design tool for achieving flexibility and code reusability.
    \end{block}
\end{frame}


\begin{frame}[plain]
    \centering
    \vfill
    {\Huge \textbf{Thank You for Your Attention!}}\\[1em]
    {\large If you have any \textit{questions}, please keep them in your \textit{mind}.}
    \vfill
\end{frame}


